\section{Метод решения}

Был создан класс \textit{LZ77}, в котором реализованы методы для кодирования и декодирования текста.
Метод \textit{LZ77::Encode} принимает на вход строку и возвращает вектор троек, 
алгоритм использует наивный поиск строки в буфере и имеет итоговую сложность $O(n^3)$.
Метод \textit{LZ77::Decode} принимает на вход вектор троек и возвращает декодированную строку, работает за $O(n)$.

\pagebreak

\section{Исходный код}

\begin{lstlisting}[language=C++]
#include <iostream>
#include <vector>

class LZ77 {
	public:
	struct Node {
		int offset;
		int size;
		char symbol;
	};
	
	static inline const char EOM = ' ';
	
	static void Encode(std::string& input, std::vector<LZ77::Node>& output) {
		int pos = 0;
		while (pos < input.size()) {
			LZ77::Node n = {0, 0, input[pos]};
			for (int i = 0; i < pos; i++) {
				if (input[i] != input[pos]) {
					continue;
				}
				int size = 1;
				while (pos + size < input.size() && input[i + size] == input[pos + size]) {
					size++;
				}
				if (size >= n.size) {
					n.size = size;
					n.offset = pos - i;
				}
			}
			if (pos + n.size >= input.size()) {
				n.symbol = LZ77::EOM;
			}
			else {
				n.symbol = input[pos + n.size];
			}
			output.push_back(n);
			pos += n.size + 1;
		}
	}
	
	static void Decode(std::vector<LZ77::Node>& input, std::string& output) {
		for (LZ77::Node& n : input) {
			if (n.offset != 0) {
				int pos = output.size();
				for(int i = 0; i < n.size; i++) {
					output.push_back(output[pos - n.offset + i]);
				}
			}
			
			if (n.symbol != EOM) {
				output.push_back(n.symbol);
			}
		}
	}
};

int main() {
	std::ios::sync_with_stdio(false);
	std::cin.tie(0);
	std::cout.tie(0);
	std::string cmd;
	std::cin >> cmd;
	
	if (cmd == "compress") {
		std::string text;
		std::cin >> text;
		std::vector<LZ77::Node> res;
		LZ77::Encode(text, res);
		for (LZ77::Node& n : res) {
			std::cout << n.offset << " " << n.size << " " << n.symbol << "\n";
		}
	}
	
	if (cmd == "decompress") {
		std::vector<LZ77::Node> code;
		LZ77::Node n;
		while (std::cin >> n.offset >> n.size) {
			if (!(std::cin >> n.symbol)) {
				n.symbol = LZ77::EOM;
			}
			code.push_back(n);
		}
		
		std::string res;
		LZ77::Decode(code, res);
		
		std::cout << res << "\n";
	}
	return 0;
}

\end{lstlisting}

\section{Консоль}
\begin{alltt}
    den@vbox:~/Документы/DA/CP$ g++ cp.cpp
    den@vbox:~/Документы/DA/CP$ ./a.out
    compress
    abracadabra
    0 0 a
    0 0 b
    0 0 r
    3 1 c
    2 1 d
    7 4  
    den@vbox:~/Документы/DA/CP$ ./a.out
    decompress
    0 0 a
    0 0 b
    0 0 r
    3 1 c
    2 1 d
    7 4
    abracadabra
\end{alltt}

