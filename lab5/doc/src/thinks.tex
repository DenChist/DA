\section{Выводы}

Выполнив пятую лабораторную работу по курсу Дискретный анализ, я познакомился с такой структурой данных как суффиксное дерево. Основное ее предназначение - поиск образцов в тексте. Поиск отличается от всеми известных алгоритмов КМП, Ахо-Корасик и др. тем, что мы предобрабатываем текст, а не образец.
Однако у суффиксных деревьев есть еще много применений. Благодаря ему мы можем за линейное время найти наибольшую общую подстроку, можем рассчитать статистику совпадений, построить суффиксный массив или найти минимальный лексикографический разрез.

