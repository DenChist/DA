\CWHeader{Лабораторная работа \textnumero 7}

\CWProblem{
При помощи метода динамического программирования разработать алгоритм решения задачи, определяемой своим вариантом; оценить время выполнения
алгоритма и объем затрачиваемой оперативной памяти. Перед выполнением задания
необходимо обосновать применимость метода динамического программирования.
Имеется натуральное число n. За один ход с ним можно произвести следующие действия: вычесть единицу, разделить на два, разделить на три.
При этом стоимость
каждой операции — текущее значение n. Стоимость преобразования — суммарная
стоимость всех операций в преобразовании. Вам необходимо с помощью последовательностей указанных операций преобразовать число n в единицу таким образом,
чтобы стоимость преобразования была наименьшей. Делить можно только нацело.\newline

\textbf{Формат ввода}\newline
В первой строке задано число 2 \leqslant n \leqslant {10^7}\newline

\textbf{Формат вывода}\newline 
Выведите на первой строке искомую наименьшую стоимость. Во второй строке должна содержаться последовательность операций. Если было произведено деление на 2 или на 3, выведите /2 (или /3). Если же было вычитание, выведите -1. Все операции выводите разделяя пробелом.
\pagebreak
