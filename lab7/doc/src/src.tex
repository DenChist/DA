\section{Описание}

\textbf{Динамическое программирование}\newline в теории управления и теории вычислительных систем — способ решения сложных задач путём разбиения их на более простые подзадачи. Он применим к задачам с оптимальной подструктурой, выглядящим как набор перекрывающихся подзадач, сложность которых чуть меньше исходной. В этом случае время вычислений, по сравнению с «наивными» методами, можно значительно сократить.\newline

Ключевая идея в динамическом программировании достаточно проста. Как правило, чтобы решить поставленную задачу, требуется решить отдельные части задачи (подзадачи), после чего объединить решения подзадач в одно общее решение. Часто многие из этих подзадач одинаковы. Подход динамического программирования состоит в том, чтобы решить каждую подзадачу только один раз, сократив тем самым количество вычислений. Это особенно полезно в случаях, когда число повторяющихся подзадач экспоненциально велико.\newline

Метод динамического программирования сверху — это простое запоминание результатов решения тех подзадач, которые могут повторно встретиться в дальнейшем. Динамическое программирование снизу включает в себя переформулирование сложной задачи в виде рекурсивной последовательности более простых подзадач.\newline
\pagebreak

\section{Исходный код}
Фрагмент кода, где применяются рекуррентные формулы:\newline


\begin{lstlisting}[language=C]
 #include <iostream>
 #include <vector>
 
 const int OP_MINUS_ONE = 0;
 const int OP_DIVIDE_TWO = 1;
 const int OP_DIVIDE_THREE = 2;
 
 int main() {
 	int n;
 	std::cin >> n;
 	std::vector<int> coast(n + 1);
 	std::vector<int> res(n + 1);
 	
 	for (int i = 2; i <= n; i++) {
 		coast[i] = coast[i - 1] + i;
 		res[i] = OP_MINUS_ONE;
 		
 		if (i % 2 == 0 && coast[i / 2] + i < coast[i]) {
 			coast[i] = coast[i / 2] + i;
 			res[i] = OP_DIVIDE_TWO;
 		}
 		if (i % 3 == 0 && coast[i / 3] + i < coast[i]) {
 			coast[i] = coast[i / 3] + i;
 			res[i] = OP_DIVIDE_THREE;
 		}
 		
 	}    
 	std::cout << coast.back() << '\n';
 	
 	for (int i = n; i > 1;) {
 		switch (res[i]) {
 			case OP_MINUS_ONE:
 			std::cout << "-1";
 			i--;
 			break;
 			case OP_DIVIDE_TWO:
 			std::cout << "/2";
 			i /= 2;
 			break;
 			case OP_DIVIDE_THREE:
 			std::cout << "/3";
 			i /= 3;
 			break;
 		}
 		if (i != 1) {
 			std::cout << " ";
 		}
 	}
 	std::cout << '\n';
 	
 	return 0;
 } 
\end{lstlisting}

\section{Консоль}
\begin{alltt}
den@vbox:~/Документы/DA/lab7$ ./a.out
82
202
-1 /3 /3 /3 /3
\end{alltt}
