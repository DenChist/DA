\section{Выводы}

Выполнив седьмую лабораторную работу по курсу «Дискретный анализ», я научи-
лась решать задачи методом динамического программирования: искать оптимальную
подструктуру и составлять рекурсивное решение, когда подпрограмма вызывает са-
ма себя, и нерекурсивное решение.\newline

Динамическое программирование стоит применять для решения задач, которые обла-
дают двумя характеристиками:\newline

1. Можно составить оптимальное решение задачи из оптимального решения ее под-
задач.\newline

2. Рекурсивный подход к решению проблемы предполагал бы многократное (не од-
нократное) решение одной и той же подпроблемы, вместо того, чтобы производить
в каждом рекурсивном цикле все новые и уникальные подпроблемы.\newline

Так, в решаемой мной задаче методом динамического программирования, можно
заметить, что имеется оптимальная подструктура решения - его можно свести к
решению подзадач меньшего размера. Действительно, минимальная стоимость р[n]
преобразования числа n равна n + min(p[n/2], p[n/3], p[n 1]). Стоимость p[n/2] или
p/n/3] считается бесконечной, если n не делится на 2 или 3 соответственно, так как
деление происходит только нацело. При построении рекурсивного решения становит-
ся ясно, что имеет место перекрытие вспомогательных подзадач, значит оптимальное
решение можно построить методом восходящего анализа. Таким образом, решение
имеет оценку $O(n)$ для времени в памяти.\newline

