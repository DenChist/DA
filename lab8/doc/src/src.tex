\section{Описание}
Данный жадный алгоритм основан на расположении длин сторон в порядке убывания и проверке начиная сверху, берется три самых больших стороны считается
площадь, если такой треугольник возможен, делаем проверку сравнивая с текущей
наибольшей площадью, если значение больше, то запоминаем. Из-за сортировки требует $O(n log_n)$ времени.

\pagebreak

\section{Исходный код}

Код: main.cpp
\begin{lstlisting}[language=C++]
#include <iostream>
#include <vector>
#include <algorithm>
#include <cmath>

double Area(int s1, int s2, int s3) {
	double p = (s1 + s2 + s3) * 0.5;
	return sqrt(p * (p - s1) * (p - s2) * (p - s3));
}

bool ValidTriangle(int s1, int s2, int s3) {
	return (s1 < (s2 + s3)) && (s2 < (s1 + s3)) && (s3 < (s1 + s2));
}

int main() {
	std::vector<int> data;
	int n = 0;
	int s = 0;
	int s1 = 0;
	int s2 = 0;
	int s3 = 0;
	double max_area = 0.0;
	double cur_area = 0.0;
	
	std::cin >> n;
	for (int i = 0; i < n; ++i) {
		std::cin >> s;
		data.push_back(s);
	}
	
	std::sort(data.begin(), data.end(), std::greater<int>());
	
	for (int i = 1; i < int(data.size() - 1); ++i) {
		if (data.size() < 3) {
			break;
		}
		if (ValidTriangle(data[i - 1], data[i], data[i + 1])) {
			cur_area = Area(data[i - 1], data[i], data[i + 1]);
			if (cur_area > max_area) {
				max_area = cur_area;
				s1 = data[i + 1];
				s2 = data[i];
				s3 = data[i - 1];
			}
		}
	}
	
	if (max_area == 0.0) {
		std::cout << 0 << '\n';
	}
	else {
		printf("%.3f\n", max_area);
		std::cout << s1 << ' ' << s2 << ' ' << s3 << '\n';
	}
	return 0;
}
\end{lstlisting}

\section{Консоль}

\begin{alltt}
den@vbox:~/Документы/DA/lab8$ ./a.out
4
1 2 3 5
0
\end{alltt}


