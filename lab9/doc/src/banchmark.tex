\section{Тест производительности}

В программе я использовал поиск в ширину для нахождения пути. Попробуем использовать поиск в глубину и сравним результаты.
Для сравнения я подготовил четыре теста с полными графами, где количество вершин равно 10, 20 и 30.

\begin{alltt}
den@vbox:~/Документы/DA/lab9$ ./bfs <test10
368639003
Time: 0.0002965 s
den@vbox:~/Документы/DA/lab9$ ./dfs <test10
368639003
Time: 0.0038175 s
den@vbox:~/Документы/DA/lab9$ ./bfs <test20
808017301
Time: 0.0005277 s
den@vbox:~/Документы/DA/lab9$ ./dfs <test20
808017301
Time: 0.0012016 s
den@vbox:~/Документы/DA/lab9$ ./bfs <test30
1252856464
Time: 0.0008176 s
den@vbox:~/Документы/DA/lab9$ ./dfs <test30
1252856464
Time: 1.36699 s
\end{alltt}

Иногда поиск в глубину значительно проигрывает по времени поиску в ширину. Дело в том, что поиск в ширину всегда находит кратчайший путь, 
на каждом шаге приближаясь к конечной точке, а поиск в глубину обходит вершины графа в случайном порядке и находит не самый оптимальный путь, совершая много лишних операций.
Однако асимптотическая сложность этих алгоритмов одинакова: $O(m + n)$, где $m$ - количество ребер графа, $n$ - количество вершин графа.

\pagebreak
