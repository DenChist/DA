\section{Описание}

Алгоритм Форда-Фалкерсона состоит из поиска в ширину и изменение весов графа в соответствии с минимальным весом ребра в найденном пути.

Алгоритм крутит поиск в ширину пока не сможет найти путь из истока в сток графа (1,n). Далее, в двудольном графе меняет веса ребер. Каждый раз находя новый путь в графе, минимльный вес является довабочным потоком в графе и оно прибавляется к общему ответу, когда алгоритм не смодет найти путь, алгоритм прекращает работу и выходит.

Скорость работы алгоритма оценивается как $O(f * |E|)$, где $f$ - кол-во найденных путей в графе, а $|E|$ - кол-во или мощность множества вершин графа. Это можно объяснить так: мы ищем путь в графе, далее для каждой вершины в пути, длина которого может быть максимум $|E|$, меняем веса за $O(1)$.

\pagebreak

\section{Исходный код}

\begin{lstlisting}[language=C++]
#include <iostream>
#include <vector>
#include <queue>
#include <unordered_map>

using namespace std;

bool BFS(vector<unordered_map<int,int>>& graph, int start, int end, vector<int>& parents) {
	parents.clear();
	parents.resize(graph.size(), -1);
	vector<bool> visited(graph.size());
	queue<int> q;
	q.push(start);
	visited[start] = true;
	
	while (!q.empty()) {
		int cur_vertex = q.front();
		q.pop();
		for (auto edge : graph[cur_vertex]) {
			int new_vertex = edge.first;
			int new_capacity = edge.second;
			if (!visited[new_vertex] && new_capacity) {
				visited[new_vertex] = true;
				q.push(new_vertex);
				parents[new_vertex] = cur_vertex;
				if (new_vertex == end) {
					return true;
				}
			}
		}
	}
	return false;
}

long long MaxFlow(vector<unordered_map<int,int>>& graph, int source, int sink) {
	long long resFlow = 0;
	vector<int> parents;
	while (BFS(graph, source, sink, parents)) {
		int flow = 1000000001;
		for (int i = sink; i != source; i = parents[i]) {
			if (i == -1) {
				throw runtime_error("Can't find a parent for not source node.");
			}
			if (graph[parents[i]][i] < flow) {
				flow = graph[parents[i]][i];
			}
		}
		resFlow += flow;
		
		for (int i = sink; i != source; i = parents[i]) {
			graph[parents[i]][i] -= flow;
			graph[i][parents[i]] += flow;
		}
	}
	return resFlow;
}

int main() {
	ios_base::sync_with_stdio(false);
	cin.tie(nullptr);
	int n, m;
	cin >> n >> m;
	vector<unordered_map<int,int>> graph(n);
	for (int i = 0; i < m; ++i) {
		int from, to, capacity;
		cin >> from >> to >> capacity;
		--from;
		--to;
		graph[from][to] = capacity;
	}
	cout << MaxFlow(graph, 0, n - 1) << "\n";
}
\end{lstlisting}



\section{Консоль}

\begin{alltt}
den@vbox:~/Документы/DA/lab9$ ./a.out
5 6
1 2 4
1 3 3
1 4 1
2 5 3
3 5 3
4 5 10
7
\end{alltt}

\pagebreak