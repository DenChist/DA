\section{Выводы}

Выполнив девятую лабораторную работу по курсу \enquote{Дискретный анализ}, я изучил способы представления и обработки графов на С++.
Граф можно представить списком ребер, матрицей смежности или списком смежности.
Изучен алгоритм Форда-Фалкерсона нахождения максимального потока. Этот алгоритм часто используется в оптимизационных задачах на практике, хотя его идея чрезмерно проста.
